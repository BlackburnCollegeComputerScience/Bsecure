% -----------------------------------------------------------------------------
% This file derived from ACM template. See acm.org
% -----------------------------------------------------------------------------
% This .tex file (and associated .cls V3.2SP) *DOES NOT* produce:
%       1) The Permission Statement
%       2) The Conference (location) Info information
%       3) The Copyright Line with ACM data
%       4) Page numbering
% -----------------------------------------------------------------------------
% It is an example which *does* use the .bib file (from which the .bbl file
% is produced).
% REMEMBER HOWEVER: After having produced the .bbl file,
% and prior to final submission,
% you need to 'insert'  your .bbl file into your source .tex file so as to provide
% ONE 'self-contained' source file.
%
% Questions regarding SIGS should be sent to
% Adrienne Griscti ---> griscti@acm.org
%
% Questions/suggestions regarding the guidelines, .tex and .cls files, etc. to
% Gerald Murray ---> murray@hq.acm.org
%
% For tracking purposes - this is V3.1SP - APRIL 2009

% The associated tex class file should be included in source directory
\documentclass{acm_proc_article-sp}

% Handle multiple authors from same affiliation
\def\sharedaffiliation{
\end{tabular}
\begin{tabular}{c}}

\begin{document}

\title{A Secure, Serverless Android Texting App Using SMS}

%
% You need the command \numberofauthors to handle the 'placement
% and alignment' of the authors beneath the title.
%
% For aesthetic reasons, we recommend 'three authors at a time'
% i.e. three 'name/affiliation blocks' be placed beneath the title.
%
% NOTE: You are NOT restricted in how many 'rows' of
% "name/affiliations" may appear. We just ask that you restrict
% the number of 'columns' to three.
%
% Because of the available 'opening page real-estate'
% we ask you to refrain from putting more than six authors
% (two rows with three columns) beneath the article title.
% More than six makes the first-page appear very cluttered indeed.
%
% Use the \alignauthor commands to handle the names
% and affiliations for an 'aesthetic maximum' of six authors.
% Add names, affiliations, addresses for
% the seventh etc. author(s) as the argument for the
% \additionalauthors command.
% These 'additional authors' will be output/set for you
% without further effort on your part as the last section in
% the body of your article BEFORE References or any Appendices.

\numberofauthors{3} 
\author{
% The command \alignauthor (no curly braces needed) should
% precede each author name, affiliation/snail-mail address and
% e-mail address. Additionally, tag each line of
% affiliation/address with \affaddr, and tag the
% e-mail address with \email.
%
% 1st. author
\alignauthor Shane Nalezyty\\
%
	% 2nd. author
\alignauthor Lucas Burdell\\
%
%3rd. author
\alignauthor Kevin Coogan\\
%
\sharedaffiliation
\affaddr{Department of Mathematics and Computer Science} \\
\affaddr{Blackburn College} \\
\affaddr{700 College Ave.} \\
\affaddr{Carlinville, IL 62626} \\
\email{\emph{\large{firstname.lastname}}@blackburn.edu}
%\and  % use '\and' if you need 'another row' of author names
} %end \author{

% There's nothing stopping you putting the seventh, eighth, etc.
% author on the opening page (as the 'third row') but we ask,
% for aesthetic reasons that you place these 'additional authors'
% in the \additional authors block, viz.

\date{30 June 2015}
% Just remember to make sure that the TOTAL number of authors
% is the number that will appear on the first page PLUS the
% number that will appear in the \additionalauthors section.

\maketitle

% We typically leave the abstract and keywords in the main document,
% and input the remaining sections from separate files.

\begin{abstract}
This paper provides a sample of a \LaTeX\ document which conforms to
the formatting guidelines for ACM SIG Proceedings.
It complements the document \textit{Author's Guide to Preparing
ACM SIG Proceedings Using \LaTeX$2_\epsilon$\ and Bib\TeX}. This
source file has been written with the intention of being
compiled under \LaTeX$2_\epsilon$\ and BibTeX.

The developers have tried to include every imaginable sort
of ``bells and whistles", such as a subtitle, footnotes on
title, subtitle and authors, as well as in the text, and
every optional component (e.g. Acknowledgments, Additional
Authors, Appendices), not to mention examples of
equations, theorems, tables and figures.

To make best use of this sample document, run it through \LaTeX\
and BibTeX, and compare this source code with the printed
output produced by the dvi file.
\end{abstract}

% A category with the (minimum) three required fields
\category{H.4}{Information Systems Applications}{Miscellaneous}
%A category including the fourth, optional field follows...
\category{D.2.8}{Software Engineering}{Metrics}[complexity measures, performance measures]

\terms{Theory}

\keywords{ACM proceedings, \LaTeX, text tagging} % NOT required for Proceedings

\section{Introduction}
Providing secure communication can be a difficult task for a variety of reasons. Common sense suggests that ideal technological solutions should be easy to use, reliable, trustworthy, and should present minimal barriers to adoption . In this paper, we present a mobile application developed on the Android platform that meets all of the above requirements to provide secure SMS communication using any device that allows standard SMS text messages and which possesses Bluetooth capabilities. Our app uses end-to-end encryption, and does not require the use of a separate server to facilitate communication or key exchange.

1. There is a real need for secure communication. NSA Edward Snowden, etc.

2. Servers are prone to attacks, and are owned by other people. Companies have been compelled to share user data by court order.

3. SMS is basic technology enabled by all communication providers. Does not require Internet access (either for server or communication itself). Since encrypted, it's okay that third party transmits and sees the data.

4. Open Source allows anyone to view our source code.

The \textit{proceedings} are the records of a conference.
ACM seeks to give these conference by-products a uniform,
high-quality appearance.  To do this, ACM has some rigid
requirements for the format of the proceedings documents: there
is a specified format (balanced  double columns), a specified
set of fonts (Arial or Helvetica and Times Roman) in
certain specified sizes (for instance, 9 point for body copy),
a specified live area (18 $\times$ 23.5 cm [7" $\times$ 9.25"]) centered on
the page, specified size of margins (1.9 cm [0.75"]) top, (2.54 cm [1"]) bottom
and (1.9 cm [.75"]) left and right; specified column width
(8.45 cm [3.33"]) and gutter size (.83 cm [.33"]).

The good news is, with only a handful of manual
settings\footnote{Two of these, the {\texttt{\char'134 numberofauthors}}
and {\texttt{\char'134 alignauthor}} commands, you have
already used; another, {\texttt{\char'134 balancecolumns}}, will
be used in your very last run of \LaTeX\ to ensure
balanced column heights on the last page.}, the \LaTeX\ document
class file handles all of this for you.

The remainder of this document is concerned with showing, in
the context of an ``actual'' document, the \LaTeX\ commands
specifically available for denoting the structure of a
proceedings paper, rather than with giving rigorous descriptions
or explanations of such commands.


\input(bsecure_background.tex)

\section{Design and Implementation}
This is a placeholder for an introductory paragraph to our design. Perhaps include a storyboard or something similar to illustrate the basic flow of the application
from the point of view of a user.

\subsection{Key Exchange}
This is a placeholder for a short paragraph or two that gives an overview of our Key Exchange section.

\subsubsection{Bluetooth Key Generation}
Explain Bluetooth sockets

Explain Bluetooth Client/Server delegation

Explain Diffie-Hellman generation using Java provided classes and objects

\subsubsection{Key Specifics}
In our source code when we use the phrase ``secret key'' we are referring to the shared secret number generated by
the Diffie-Hellman process. This is not to be confused with public-key encryption, as these keys are used symmetrically.

The keys are used to encrypt and decrypt messages using the AES-256 cipher in CBC

Explain how the use of the word ``key'' means the secret number generated from Diffie-Hellman Key Exchange, not public key encryption

AES-256 Cipher in CBC mode (With a static IV currently, IV list is planned feature)

Encoded to Base64 String for storage and handling purposes (Byte arrays can't be saved in SQLite directly, Strings are easier to move around activities (The functionality for this was already built-into Android SDK))

\subsection{Information Storage and Control}
This is a placeholder for a short paragraph or two that provides an overview of our Information Storage and Control systems.

\subsubsection{BSecure Databases}
Explain the sequence number idea (Consumer/Producer control scheme for storage)

Explain key expiration (Each time key is ``used, a value is decremented)

Explain how we use the Android Contact\_ID (The primary key of the Android contact DB)

We stored all our important values for maintaining each BSecure associated contact
into a database. The values stored in this database table are used to decide what key to use for that
contact.

Refer to Table \ref{table:contacttable} for the row layout of our SQL database.

\begin{table*}
\centering
\caption{Contact Table Design}
\label{table:contacttable}
\begin{tabular}{|c|c|c|c|c|c|} \hline
Contact\_ID&CURRENT\_SEQUENCE\_NUM&MAX\_SEQUENCE\_NUM&TOTAL\_KEYS&USES\_LEFT&USES\_MAX\\ \hline\end{tabular}
\end{table*}

\subsubsection{User Settings}
Elaborate on settings available to the user

Option to force a current key for a contact to expire

Option to force all keys with a contact to expire

Option to force all keys to expire with all contacts

If the user lacks keys with a contact they can still send unencrypted messages

User chooses a maximum and minimum number of times a key is to be used. During Bluetooth exchange
these values are sent to the other device and a decision is made on which mutual value to use for the maximum key use setting.

\subsection{Use of Text Message Headers}
This is a placeholder for an introductory paragraph on using headers in our text messages.

\subsubsection{BSecure Message Header}
Explain the use of a header in the text message to indicate that the message is from a BSecure application

We appended a header to the beginning of encrypted text messages to flag to the receiving application that
the incomming message is a BSecure encrypted message.

\subsubsection{Expire Current Key Header}
Explain the use of a header to notify that the contact has expired their key early (So keys remain synchronized across both devices).

\subsubsection{Expire All Keys Header}
Explain the use of a header to notify that the contact has expired ALL keys early.




\input{bsecure_concl.tex}

%ACKNOWLEDGMENTS are optional
\section{Acknowledgments}
This work was made possible by a Summer Research grant from Blackburn College. The authors
wish to thank the administration and the proposal reviewers for their consideration.

We would also like to express our gratitude to Traci Kamp. As a Computer Science senior, Traci
was instrumental in getting the department's Android development program off the ground, and
was responsible for the early stages of the basic framework of our texting app.



%
% The following two commands are all you need in the
% initial runs of your .tex file to
% produce the bibliography for the citations in your paper.
\bibliographystyle{abbrv}
\bibliography{bsecure}  % bsecure.bib is the name of the Bibliography in this case
% You must have a proper ".bib" file
%  and remember to run:
% latex bibtex latex latex
% to resolve all references
%
% ACM may or may not need 'a single self-contained file'!
% if so, then files should be merged for final submission.
%
% If no appendices, following line manually goes into .bbl at point 
% that balances columns
%\balancecolumns

%APPENDICES are optional
\appendix
%Appendix A
\section{Headings in Appendices}
The rules about hierarchical headings discussed above for
the body of the article are different in the appendices.
In the \textbf{appendix} environment, the command
\textbf{section} is used to
indicate the start of each Appendix, with alphabetic order
designation (i.e. the first is A, the second B, etc.) and
a title (if you include one).  So, if you need
hierarchical structure
\textit{within} an Appendix, start with \textbf{subsection} as the
highest level. Here is an outline of the body of this
document in Appendix-appropriate form:
\subsection{Introduction}
\subsection{The Body of the Paper}
\subsubsection{Type Changes and  Special Characters}
\subsubsection{Math Equations}
\paragraph{Inline (In-text) Equations}
\paragraph{Display Equations}
\subsubsection{Citations}
\subsubsection{Tables}
\subsubsection{Figures}
\subsubsection{Theorem-like Constructs}
\subsubsection*{A Caveat for the \TeX\ Expert}
\subsection{Conclusions}
\subsection{Acknowledgments}
\subsection{Additional Authors}
This section is inserted by \LaTeX; you do not insert it.
You just add the names and information in the
\texttt{{\char'134}additionalauthors} command at the start
of the document.
\subsection{References}
Generated by bibtex from your ~.bib file.  Run latex,
then bibtex, then latex twice (to resolve references)
to create the ~.bbl file.  Insert that ~.bbl file into
the .tex source file and comment out
the command \texttt{{\char'134}thebibliography}.
% This next section command marks the start of
% Appendix B, and does not continue the present hierarchy
\section{More Help for the Hardy}
The acm\_proc\_article-sp document class file itself is chock-full of succinct
and helpful comments.  If you consider yourself a moderately
experienced to expert user of \LaTeX, you may find reading
it useful but please remember not to change it.
\balancecolumns
% That's all folks!
\end{document}
