\section{RELATED WORK}
Short Message Service security has quickly become a topic of discussion in recent years.
SMS Sec \cite{SMSSec} developed in 2008 and PK-SIM \cite{PkSim} in 2009 began to try and optimise SMS encryption.
Both started developing a two phase approach to SMS encryption.
First a handshake is performed using asymmetric cryptography which occurs only a few times, and a more efficient symmetric handshake which is used frequently.
With PK-SIM \cite{PkSim} a additional trusted third party certificate authority is used to provide public key certification.

A study done in 2010 named SMS Security \cite{SmsSecurity} concluded that large key algorithms for SMS encryption are not suitable due to the small memory capacity of mobile phones.
In this study they evaluated RSA, ELGamal and ECDSA algorithms which are known for having keys longer then 1024 bits.
Secure SMS \cite{SecureSms} aimed to develop on pre existing protocols by evaluating security exploits, and removing the significant overhead created with public key encryption.
Secure SMS concluded that symmetric key algorithm performance was a thousand times faster then asymmetric key algorithms.
It must be noted that Secure SMS preferred efficiency to eschew security in the investigation, and they recommend careful independent investigation when selecting encryption algorithms.
By using AES256 symmetric key encryption we are not concerned with the overhead created when requiring large keys.

There are existing projects that provide free and paid encrypted communication such as Bleep \cite{bleep} or Threema \cite{threema}.
However these solutions rely on passing data connections instead of pre-existing SMS communication.
Neither Bleep or Threema have publicly released their source code.
Threema also passes all messages through their servers.

